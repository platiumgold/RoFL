\documentclass[12pt,a4paper]{article}

\usepackage[utf8]{inputenc}
\usepackage[T2A]{fontenc}
\usepackage[russian]{babel}

\usepackage{amsmath}
\usepackage{amssymb}

\usepackage{graphicx}
\usepackage{float}

\usepackage{geometry}
\geometry{top=2cm, bottom=2cm, left=2cm, right=2cm}

\usepackage{xcolor}

\usepackage{csvsimple}

\title{lab2 Амбросий Николай Евгеньевич ИУ9-51Б}
\author{}
\date{}

\begin{document}
	
	\maketitle
	
	\section{минимальный ДКА}
	Имеем регулярное выражение:
	
	$$ ((a^*b^*c^*)^*ab(a^*b^*c^*)^*bc(a|b|c)^*)\mid((a|b|c)^*bc(a^*b^*c^*)^*ab(a|bc|cc|bb)^*)\mid abc $$
	
	
	Преобразуем его в вид, удобный для парсинга. Явно обозначим конкатенацию символом «$\cdot$»:
	$$
		\begin{aligned}
			((a^*\cdot b^*\cdot c^*)^* \cdot a \cdot b \cdot (a^*\cdot b^*\cdot c^*)^* \cdot b \cdot c \cdot (a|b|c)^*) \mid \\
			((a|b|c)^* \cdot b \cdot c \cdot (a^*\cdot b^*\cdot c^*)^* \cdot a \cdot b \cdot (a|b \cdot c|c \cdot c|b \cdot b)^*) \mid \\
			a \cdot b \cdot c
		\end{aligned}
	$$
	
	Преобразуем полученное выражение в Обратную Польскую Нотацию:
	
		$ a*b*.c*.*a.b.a*b*.c*.*.b.c.ab|c|*.ab|c|*b.c.a*b*.c*.*.a.b.abc.|cc.|bb.|*.|ab.c.| $
	
	Применим метод Томпсона к обратной польской записи и НКА.
	
	\begin{figure}[H]
		\centering
		\includegraphics[width=0.8\textwidth]{automatas/Thompson.png} 
	\end{figure}
	
	Выполним детерминизацию полученного НКА
	
	\begin{figure}[H]
		\centering
		\includegraphics[width=0.8\textwidth]{automatas/DFA.png} 
	\end{figure}
	
	A затем минимизацию полученного ДКА.
	
	\begin{figure}[H]
		\centering
		\includegraphics[width=0.8\textwidth]{automatas/min_DFA.png} 
	\end{figure}
	

	\csvautotabular{automatas/min_table.csv}

	
	\section{малый НКА}
	Упростим исходное регулярное выражение.
	
	Заметим, что конструкция $(a^*b^*c^*)^*$ при условии алфавита $\Sigma = \{a, b, c\}$ эквивалентна $(a|b|c)^*$.
	
	Упрощенное выражение принимает вид:
	$$ ((a|b|c)^* ab (a|b|c)^* bc (a|b|c)^*) \mid ((a|b|c)^* bc (a|b|c)^* ab (a|bc|cc|bb)^*) \mid abc $$
	
	построим по нему малый НКА
	
	\begin{figure}[H]
		\centering
		\includegraphics[width=0.8\textwidth]{automatas/mini_nfa.png} 
		\label{fig:rel_diagram}
	\end{figure}
	Получилось выделить 7 различных КЭ. последний суффикс abbc никаким префиксом из языка не вывести и минус на нем не получить. Итак, в минимальном НКА как минимум 7 состояний
	
	
	\csvautotabular{automatas/nfa_table.csv}
	
	\section{Расширенные регулярные выражения}
	Вынесем проверку алфавита $\{a, b, c\}$ в lookahead. Тогда конструкции $(a|b|c)^*$ можно заменить на $.^*$ .
	
	$ \textasciicircum (?=[abc]^*\$)(.^*ab.^*bc.^*|.^*bc.^*ab(a|bc|cc|bb)^*|abc)\$ $
	\section{ПКА}
	
	Наш язык L представляет собой объединение трех языков: $L = L_1 \cup L_2 \cup L_3$.
	
	Соответственно И недетерминизм надо ввести в каком-нибудь $L_i$.
	У $L_1$ и $L_3$ структура линейна и проста, вводить распараллеливание негде
	
	Остается $L_2$. В нем префиксная часть $.^* bc .^* ab$ также линейна. Единственное место, допускающее параллельную обработку — это суффикс:
	$ (a \mid bc \mid cc \mid bb)^* $
	
	Нас интересуют составные подслова bc и bb. Можно разделить обработку на два потока:
	\begin{enumerate}
		\item Поток 1: После нахождения b просто продолжает чтение, ожидая новые символы согласно структуре.
		\item Поток 2: Проверяет, является ли текущая последовательность допустимым подсловом (bc или bb) в контексте суффикса.
	\end{enumerate}
	
	\begin{figure}[H]
		\centering
		\includegraphics[width=0.8\textwidth]{automatas/funny_AFA.png} 
	\end{figure}
	Нижняя оценка числа состояний ПКА - 3.
	
	\csvautotabular{automatas/afa_table.csv}
	
\end{document}